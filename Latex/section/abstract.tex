Trên thị trường hiện nay, vàng (gold), bạch kim (platinum) và palladium là các kim loại quý hiếm, có giá trị cao và nhu cầu về các kim loại này không bao giờ lỗi thời. Xu hướng tỷ giá của kim loại quý cho thấy đây là một trong những phương án đầu tư tốt nhất hiện nay. Vì vậy, mối quan tâm đến việc ứng dụng các phương pháp thống kê, học máy và học sâu để hiểu rõ và dự đoán xu hướng của tỷ giá này có ý nghĩa như thế nào đối với nền kinh tế đất nước ngày càng cao. Bài báo sẽ nghiên cứu các ý tưởng dự đoán tỷ giá vàng (Gold), bạch kim (Ơlatinum) và Palladium bằng các mô hình dự báo: AMIRA, RNN, GRU, LSTM, ETS, LN, SVM, Random Forest, TimesNet, Autoformer. Nghiên cứu áp dụng 10 mô hình dự báo khác nhau để tìm ra mô hình nào hoạt động tốt nhất trên tập dữ liệu có sẵn. Bộ dữ liệu về giá kim loại quý được chia theo tập train:test với 3 tỷ lệ là 6:4, 7:3, 8:2. Sau đó  thực hiện so sánh hiệu suất của các mô hình dựa trên ba độ đo: MSE, RMSE, MAPE. Cuối cùng, thực hiện dự đoán giá của kim loại quý trong 30,60 và 90 ngày tiếp theo đối với tất cả mô hình. Kết quả cho thấy mô hình RF, SVR hoặc Autoformer có hiệu suất ổn định nhất và có thể giúp cải thiện dự đoán. Từ đó, đưa ra các quyết định hiệu quả trong thị trường thực tế và giúp ích cho việc phát triển và đầu tư vào các kim loại quý.