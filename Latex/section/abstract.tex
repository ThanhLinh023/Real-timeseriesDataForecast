Trên thị trường hiện nay, vàng (Gold), bạch kim (Platinum) và Palladium là các kim loại quý hiếm, có giá trị cao và nhu cầu về các kim loại này không bao giờ giảm. Xu hướng tỷ giá của kim loại quý cho thấy đây là một trong những phương án đầu tư tốt nhất hiện nay. Vì vậy, mối quan tâm đến việc sử dụng các mô hình thống kê, học máy và học sâu để hiểu rõ và dự báo xu hướng của tỷ giá này có ý nghĩa rất lớn đối với nền kinh tế đất nước. Bài báo sẽ nghiên cứu các ý tưởng dự báo tỷ giá vàng (Gold), bạch kim (Platinum) và Palladium bằng các mô hình: Linear Regression (LR), Exponential Smoothing Trend (ETS), ARIMA, Random Forest (RF), Support Vector Regression (SVR), Recurrent Neural Network (RNN), Gated Recurrent Unit  (GRU), Long Short Term Memory (LSTM), Timesnet, Autoformer. Nghiên cứu áp dụng 10 mô hình dự báo khác nhau để tìm ra mô hình nào hoạt động tốt nhất trên tập dữ liệu có sẵn. Bộ dữ liệu về giá kim loại quý được chia theo tập train:test với 3 tỷ lệ là 6:4, 7:3, 8:2. Sau đó thực hiện so sánh hiệu suất của các mô hình dựa trên ba độ đo: MSE, RMSE, MAPE. Cuối cùng, thực hiện dự báo giá của kim loại quý trong 30, 60 và 90 ngày tiếp theo đối với tất cả mô hình. Kết quả cho thấy mô hình TimesNet, RNN và GRU có hiệu suất ổn định nhất và có thể giúp cải thiện dự báo. Từ đó, đưa ra các quyết định hiệu quả trong thị trường thực tế và giúp ích cho việc phát triển và đầu tư vào các kim loại quý.