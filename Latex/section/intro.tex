Hiện tại, trong bối cảnh nền kinh tế thế giới nói chung và Việt Nam nói riêng đang diễn biến 1 cách phức tạp và khó dự đoán kể từ sau đại dịch COVID-19. Các lĩnh vực chủ đạo của nền kinh tế ít nhiều bị biến động, do đó giá của các kim loại quý như vàng (Gold), bạch kim (Platinum) và Palladium cũng bị biến động theo nền kinh tế hiện tại. Và việc dự báo giá cũng là 1 vấn đề quan trọng trong lĩnh vực tài chính và đầu tư. Việc dự báo chính xác giá của các kim loại quý này có thể giúp các nhà đầu tư, doanh nghiệp và các tổ chức khác đưa ra quyết định sáng suốt về việc mua bán, nắm giữ hoặc đầu tư vào chúng. Nắm bắt được vấn đề hiện hữu, nhóm chúng tôi đã thực hiện nghiên cứu này với các mô hình dự báo chuỗi thời gian để có thể đưa ra được các dự báo với tỉ lệ chính xác cao nhằm giúp ích cho việc phát triển và đầu tư.

Để dự báo giá của các kim loại quý, có khá nhiều thuật toán và kỹ thuật hỗ trợ cho việc này. Và trong nghiên cứu này, chúng tôi sẽ sử dụng 10 mô hình để áp dụng cho việc dự báo bao gồm: Linear Regression (LR), Exponential Smoothing Trend (ETS), ARIMA, Random Forest (RF), Support Vector Regression (SVR), Recurrent Neural Network (RNN), Gated Recurrent Unit  (GRU), Long Short Term Memory (LSTM), Timesnet, Autoformer. Những mô hình này được áp dụng và sẽ dự báo giá của kim loại quý trong 30, 60 và 90 ngày tới.

Dựa vào kết quả dự báo, chúng tôi có thể phân tích, đánh giá và kiểm tra độ chính xác của các giá dự báo bằng cách so sánh với giá thực tế. Điều này giúp tinh chỉnh các mô hình và phương pháp dự báo để đạt độ chính xác cao hơn. Ngoài ra, các kết quả dự báo này có thể sẽ giúp cho các nhà đầu tư, doanh nghiệp và các tổ chức tài chính để họ sử dụng trong việc lập kế hoạch và đưa ra các quyết định đầu tư.