Hiện tại, trong bối cảnh nền kinh tế thế giới nói chung và Việt Nam nói riêng đang diễn biến 1 cách phức tạp và khó dự đoán kể từ sau đại dịch COVID-19. Các lĩnh vực chủ đạo của nền kinh tế ít nhiều bị biến động do đó việc giá của các kim loại quý giá như Vàng (Gold), Bạch kim (Platinum), Palladium cũng bị biến động theo nền kinh tế hiện tại. Và việc dự đoán giá cũng là 1 vấn đề quan trọng trong lĩnh vực tài chính và đầu tư. Việc dự đoán chính xác giá của các kim loại quý này có thể giúp các nhà đầu tư, doanh nghiệp và các tổ chức khác đưa ra quyết định sáng suốt về việc mua bán, nắm giữ hoặc đầu tư vào chúng. Nắm bắt được vấn đề hiện hữu, nhóm chúng tôi đã thực hiện các nghiên cứu trên các mô hình dự đoán chuỗi thời gian để có thể đưa ra được các dự báo với tỉ lệ chính xác cao nhằm giúp ích cho việc phát triển và đầu tư.\newline
Tại sao chúng ta cần thực hiện dự án này: Đầu tiên, giúp quản lý rủi ro - Biến động giá kim loại quý có thể gây ra rủi ro tài chính lớn. Dự đoán chính xác giúp các nhà đầu tư và doanh nghiệp có chiến lược quản lý rủi ro hiệu quả hơn. Thứ hai, giúp tối ưu hóa lợi nhuận - Bằng cách dự đoán xu hướng giá, các nhà đầu tư có thể tối ưu hóa lợi nhuận thông qua việc chọn thời điểm mua vào hoặc bán ra hợp lý. Thứ ba, hỗ trợ quyết định đầu tư - Các dự báo chính xác cung cấp thông tin quan trọng giúp các nhà đầu tư và doanh nghiệp đưa ra quyết định đầu tư có cơ sở khoa học và dữ liệu thực tế.\newline
Để dự báo giá của các kim loại, có khá nhiều thuật toán và kỹ thuật hỗ trợ cho việc này. Và trong dự án này nhóm em sẽ nghiên cứu 10 mô hình để áp dụng cho việc dự báo bao gồm: Linear Regression, ETS, ARIMA, Random Forest, RNN, SVM, LSTM, Autoformer, GRU, TimesNet. Những mô hình này được áp dụng và sẽ dự báo giá của kim loại trong 30, 60 và 90 ngày tới, sau khi kết thúc dự án. \newline
Dự định sau khi có giá dự báo: Có thể giúp phân tích, đánh giá và kiểm tra độ chính xác của các dự báo bằng cách so sánh với giá thực tế. Điều này giúp tinh chỉnh các mô hình và phương pháp dự đoán để đạt độ chính xác cao hơn. Ứng dụng vào thực tế: Cung cấp các dự báo cho các nhà đầu tư, doanh nghiệp và các tổ chức tài chính để họ sử dụng trong việc lập kế hoạch và quyết định đầu tư. Xuất bản kết quả nghiên cứu: Chia sẻ kết quả nghiên cứu và các mô hình dự đoán với cộng đồng học thuật và các chuyên gia trong ngành thông qua các báo cáo khoa học, bài báo và hội thảo.