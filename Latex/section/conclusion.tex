Trong nghiên cứu này, chúng tôi đã thực hiện việc áp dụng các mô hình thống kê, học máy và học sâu để dự đoán giá kim loại quý. Chúng tôi sử dụng các mô hình như LR, ETS, AMIRA, RF, SVM, RNN, GRU, LSTM, TimesNet và Autoformer trên ba bộ dữ liệu khác nhau nhằm dự đoán giá của kim loại quý. Qua đó, chúng tôi đánh giá và so sánh hiệu suất của từng mô hình cho thấy RF, SVM và Autoformer có kết quả khá tốt. Từ đó, có thể giúp cải thiện dự đoán và đưa ra quyết định hiệu quả trong thị trường thực tế, đặc biệt là đối với các thị trường kim loại quý như Gold, Platinum và Palladium.

Dù đạt được một số kết quả tích cực, nghiên cứu của chúng tôi cũng gặp phải một số thách thức. Một trong những thách thức đó là sự phức tạp và biến động của thị trường kinh tế, điều này làm cho việc dự đoán giá kim loại trở nên khó khăn hơn.

Trong tương lai, chúng tôi dự định sẽ tiếp tục nghiên cứu và áp dụng các kỹ thuật tinh chỉnh mô hình để nâng cao hiệu quả dự đoán. Ngoài ra, chúng tôi sẽ xem xét việc sử dụng các mô hình mới nhất và phát triển phương pháp kết hợp giữa các mô hình khác nhau để tăng độ chính xác và độ tin cậy của dự đoán. Chúng tôi cũng sẽ mở rộng phạm vi nghiên cứu bằng cách sử dụng thêm nhiều dữ liệu từ các thị trường kim loại quý khác nhau nhằm đạt được sự dự đoán chính xác hơn.