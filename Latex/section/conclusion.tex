Trong bài báo này, chúng tôi đã thực hiện việc áp dụng các mô hình thống kê, học máy và học sâu để dự báo giá kim loại quý. Chúng tôi đã sử dụng các mô hình: Linear Regression (LR), Exponential Smoothing Trend (ETS), ARIMA, Random Forest (RF), Support Vector Regression (SVR), Recurrent Neural Network (RNN), Gated Recurrent Unit  (GRU), Long Short Term Memory (LSTM), Timesnet, Autoformer trên ba bộ dữ liệu khác nhau để dự báo giá kim loại quý. Qua đó, chúng tôi đánh giá và so sánh hiệu suất của từng mô hình cho thấy TimesNet, RNN và GRU có kết quả khá tốt. Điều này chỉ ra tiềm năng của các mô hình dựa trên mạng nơ-ron sâu trong lĩnh vực dự báo giá kim loại quý. 

Dù đạt được một số kết quả tích cực, nghiên cứu của chúng tôi cũng gặp phải một số thách thức. Một trong những thách thức đó là sự phức tạp và biến động của thị trường kinh tế, điều này làm cho việc dự báo giá kim loại trở nên khó khăn hơn.

Trong tương lai, chúng tôi dự định sẽ tiếp tục nghiên cứu và áp dụng các kỹ thuật tinh chỉnh mô hình để nâng cao hiệu quả dự báo. Ngoài ra, các mô hình nơ-ron sâu như TimesNet, RNN và GRU không chỉ giúp cải thiện độ chính xác của dự báo mà còn mở ra những cơ hội mới trong việc áp dụng trí tuệ nhân tạo để phân tích và dự báo các biến động phức tạp trong thị trường kim loại quý. Sự kết hợp giữa khả năng học hỏi sâu và khả năng xử lý dữ liệu chuỗi của các mạng nơ-ron này làm cho chúng trở thành công cụ quan trọng và hiệu quả trong nghiên cứu và thực tiễn đầu tư vàng (Gold), bạch kim (Platinum) và Palladium. Bên cạnh đó, chúng tôi sẽ xem xét việc sử dụng các mô hình mới nhất và phát triển phương pháp kết hợp giữa các mô hình khác nhau để tăng độ chính xác và độ tin cậy của dự báo. Chúng tôi cũng sẽ mở rộng phạm vi nghiên cứu bằng cách sử dụng thêm nhiều dữ liệu từ các thị trường kim loại quý khác nhau nhằm đạt được sự dự báo chính xác hơn.