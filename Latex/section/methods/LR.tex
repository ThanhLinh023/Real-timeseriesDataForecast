Linear Regression là một phương pháp dùng để xác định mối quan hệ giữa một biến phụ thuộc và một hoặc nhiều biến độc lập. Mối quan hệ này được mô tả bằng một phương trình tuyến tính, trong đó biến phụ thuộc được thể hiện là một hàm tuyến tính của các biến độc lập.\\
Khi chỉ có một biến độc lập, chúng ta gọi là hồi quy tuyến tính đơn giản (Simple Linear Regression). Trong trường hợp có nhiều biến độc lập, ta gọi là hồi quy tuyến tính đa biến (Multiple Linear Legression).\\
Simple Linear Regression được mô tả qua công thức:\\
\[
y=\beta_{0}+\beta_{1}x+\varepsilon
\]
Trong đó:\\
\indent\textbullet\ \(y\): biến phụ thuộc (dependent variable) cần dự đoán.\\
\indent\textbullet\ \(x\): biến độc lập (independent variable) được sử dụng để dự đoán giá trị của y.\\
\indent\textbullet\ \(\beta_{0}\): hệ số góc (intercept) của đường hồi quy, đại diện cho giá trị dự đoán của y khi x=0.\\
\indent\textbullet\ \(\beta_{1}\): hệ số hồi quy (regression coefficient), đại diện cho mức độ thay đổi của y dựa trên mỗi đơn vị thay đổi của x.\\
\indent\textbullet\ \(\varepsilon\): lỗi ngẫu nhiên (random error), biểu thị sự không thể tránh khỏi của mô hình trong việc mô phỏng dữ liệu thực tế.\\