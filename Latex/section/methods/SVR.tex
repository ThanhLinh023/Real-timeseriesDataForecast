SVR (Support Vector Regression) là một loại máy vector hỗ trợ SVM được sử dụng cho các nhiệm vụ phân tích hồi quy. SVR sẽ tìm hàm dự đoán tốt nhất cho giá trị đầu ra liên tục dựa trên giá trị đầu vào cho trước. SVR có thể sử dụng cả các kernel tuyến tính và phi tuyến. Một kernel tuyến tính là tích vô hướng đơn giản giữa hai vector đầu vào, trong khi một kernel phi tuyến là một hàm phức tạp hơn có thể nắm bắt các mẫu phức tạp hơn trong dữ liệu. Sự lựa chọn của kernel phụ thuộc vào các đặc tính của dữ liệu và độ phức tạp của nhiệm vụ.

Không giống như các mô hình hồi quy truyền thống, SVR tập trung vào việc giảm thiểu lỗi dự đoán thay vì khớp dữ liệu một cách chính xác. Để đạt được điều này bằng cách tìm một siêu phẳng tối ưu tối ưu hóa khoảng cách hay biên, giữa các giá trị dự đoán và các điểm dữ liệu thực tế. SVR đạt được sự cân bằng giữa tính đơn giản và tính linh hoạt bằng cách cho phép một mức đúng sai nhất định, hay biên độ lỗi, xung quanh các giá trị dự đoán. Bây giờ chúng ta hãy định nghĩa một loại hàm mất mát khác gọi là hàm mất mát E-insensitive trong SVR.

\[
L(y, F(x_i, \hat{w})) = \max(0, y - F(x_i, \hat{w}) - \varepsilon)
\]\\
Trong đó:\\
    \indent\textbullet\ \(L(y,F(x_i,\hat{w})\): hàm lỗi.\\
    \indent\textbullet\ \(y\): giá trị thực.\\
    \indent\textbullet\ \(F(x_i,\hat{w})\): sai số ngẫu nhiên.\\
    \indent\textbullet\ \(\varepsilon\): tham số điều chỉnh cho phép sai lệch.\\
Dưới đây là một số hàm kernel có thể sử dụng trong SVR
\begin{table}[htbp]
  \centering
\begin{tabular}{|c|c|}
    \hline
     Kernel& Hàm\\ \hline
     Linear &  $f(X1,X2)=X1^TX2$\\ \hline
     Polynomial & $f(X1,X2)=(X1^TX2 +1)^d$ \\ \hline
     Sigmoid &  $f(X1,X2)=\tanh(\alpha x^{T}y+x)$\\ \hline
     RBF &  $f(X1,X2)=e^{\frac{-{\mid\mid x1-x2 \mid\mid}^2}{2\sigma^2}}$\\ \hline
\end{tabular}
\end{table}\\
\textbf{Linear}: Đây là kernel đơn giản nhất và cơ bản nhất trong SVR. Nó thể hiện chiều tuyến tính của dữ liệu đầu vào không gian tuyến tính.\\
\textbf{Polynomial}: Kernel này biến đổi dữ liệu đầu vào không gian đa thức, được xác định bởi tham số bậc và hệ số độ lệch.\\
\textbf{Sigmoid}: Kernel này biến đổi dữ liệu đầu vào thành không gian phi tuyến bằng cách sử dụng hàm sigmoid. Nó có thể được sử dụng cho các bài toán có dữ liệu nhị phân.\\
\textbf{RBF}: Kernel này sử dụng hàm RBF để biến đổi dữ liệu đầu vào thành không gian phi tuyến. Hàm RBF được xác định bởi tham số độ dốc.\\