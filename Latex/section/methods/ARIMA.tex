Mô hình ARIMA (Autoregressive Integrated Moving Average) được giới thiệu lần đầu bởi George Box và Gwilym Jenkins vào đầu những năm 1970. Từ đó đến nay, mô hình này đã trở thành một công cụ cơ bản trong phân tích và dự báo chuỗi thời gian. Mô hình ARIMA thường được biểu thị với bộ tham số là (p, d, q).\\
\textbf{Auto-Regressive (AR):} Sử dụng một tổ hợp tuyến tính của các giá trị quá khứ của biến. Một mô hình tự hồi quy (AR) bậc p có thể được viết là:
\[
Y_{t}=\phi_{1}Y_{t-1}+\phi_{2}Y_{t-2}+\cdots+\phi_{p}Y_{t-p}+\varepsilon_{t}
\]
Trong đó:\\
    \indent\textbullet\ \(Y_{t}\) là giá trị hiện tại.\\
    \indent\textbullet\ \(\phi_{1},\phi_{2},\cdots,\phi_{p}\) là các tham số của mô hình.\\
    \indent\textbullet\ \(\varepsilon_{t}\) là sai số ngẫu nhiên.\\
\textbf{Integrated (I)}: Ám chỉ việc đạo hàm của dữ liệu chuỗi thời gian.\\
\textbf{Moving Average (MA)}: Sử dụng các sai số dự báo quá khứ trong một mô hình tương tự hồi quy. "q" là số lượng giá trị sai số trước đó được xem xét cho dự báo.
\[
Y_{t}=c+\varepsilon_{t}+\theta_{1}\varepsilon_{t-1}+\theta_{2}\varepsilon_{t-2}+ \cdots+ \theta_{q}\varepsilon_{t-q}
\]
Trong đó:\\
    \indent\textbullet\ \(Y_{t}\) là giá trị hiện tại.\\
    \indent\textbullet\ \(c\) là hằng số.\\
    \indent\textbullet\ \(\theta_{1},\theta_{2},\cdots,\theta_{p}\) là các tham số của mô hình.\\
    \indent\textbullet\ \(\varepsilon_{t}\) là sai số ngẫu nhiên.