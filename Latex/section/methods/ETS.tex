ETS (Exponential Smoothing hay Error, Trend, Seasonal) là một phương thức dự báo chuỗi thời gian, mỗi mô hình được xác định bởi ba thành phần:\\
\indent\textbullet\ \textbf{Error} (Sai số): Đây là những biến động ngẫu nhiên, không thể dự đoán trước trong chuỗi thời gian. Mô hình ETS có thể xử lý sai số theo hai cách: cộng dồn (cộng sai số vào dự báo) hoặc nhân lên (nhân sai số với dự báo).\\
\indent\textbullet\ \textbf{Trend} (Xu hướng): Thể hiện hướng đi chung của chuỗi thời gian (tăng, giảm hay không có xu hướng). Mô hình ETS có thể xem xét các xu hướng khác nhau như không có xu hướng, xu hướng tăng đều (tuyến tính) hoặc xu hướng tăng nhanh dần (dùng hàm mũ).\\
\indent\textbullet\ \textbf{Seasonal} (Tính mùa vụ): Đây là những biến động lặp lại theo chu kỳ. Mô hình ETS cũng có thể xử lý tính mùa vụ theo hai cách tương tự như sai số là cộng dồn hoặc nhân lên. \\
Trong bài này, nhóm đã sử dụng phương pháp tìm kiếm lưới (grid search) để xác định tổ hợp tham số tối ưu (bao gồm: sai số, xu hướng, tính mùa vụ, chu kỳ và xu hướng giảm dần) cho mô hình ETS. Mục tiêu là tìm ra mô hình có sai số dự báo thấp nhất (đánh giá bằng MSE) trên tập dữ liệu huấn luyện, sau đó áp dụng mô hình này để dự báo 30, 60, 90 ngày cho cả ba tập dữ liệu. 
